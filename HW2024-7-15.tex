\documentclass[UTF8]{ctexart}

\usepackage{amsmath}
\usepackage{cases}
\usepackage{cite}
\usepackage{graphicx}
\usepackage[margin=1in]{geometry}
\geometry{a4paper}
\usepackage{fancyhdr}
\pagestyle{fancy}
\fancyhf{}
\newtheorem{thm}{Theorem}[section]
\newtheorem{defn}[thm]{Definition}
\newtheorem{lem}[thm]{Lemma}
\newtheorem{prop}[thm]{Proposition}
\newtheorem{cor}[thm]{Corollary}
\newtheorem{rem}[thm]{Remark}
\newtheorem{exa}[thm]{Example}
\newtheorem{exe}[thm]{Exercise}
\newtheorem{for}[thm]{Formula}

\title{Homework2024.7.15}
\author{Rerhythm}
\date{2024年7月15日}
\pagenumbering{arabic}

\begin{document}

\fancyhead[L]{Rerhythm}
\fancyhead[C]{Homework2024.7.15}
\fancyfoot[C]{\thepage}

\maketitle
\section{Introduction}
早上好,或者中午、下午、晚上好,昨天我们学习了实数、方程、乘方的运算,并且定义了一种新的运算:开$n$次方。用这些东西,我们建立了一项重要的本领:指数运算。今天我们先练练这一项,请在7月15日晚上睡觉前完成发给我,最迟不要迟于7月16日晚上。
\section{Review}
\begin{for}[指数的运算法则]
$m$,$n$是正整数,$a$,$b$是任意非0实数,我们有
    \[a^m\cdot a^n=a^{m+n}\]
    \[(a^m)^n=a^{mn}\]
    \[a^{-m}=\frac{1}{a^m}\]
    \[a^{\frac{1}{m}}=\sqrt[m]{a}\]
    \[(ab)^m=a^mb^m\]
\end{for}
\begin{rem}
    通过上述五条运算性质(或定义),我们可以自然的得到$m$,$n$是有理数情况下的完整运算法则,这些留给了你作为思考题,尝试进行推导。
\end{rem}
\section{Exercise}
\begin{exe}[概念回顾]
    用方程叙述一个数$a$的$m$($m$为正整数)次方根的定义,并说明什么情况下$a$的$m$次方根是个实数(即“有定义”)。

    hint:方程有实数解。
\end{exe}
\begin{exe}[计算练习]求值:

    (1)$8^{\frac{2}{3}}$;
    
    (2)$25^{-\frac{1}{2}}$;
    
    (3)$(\frac{1}{2})^{-5}$;
    
    (4)$(\frac{16}{81})^{-\frac{3}{4}}$
\end{exe}
\begin{exe}[计算]
    \[(a^{-2}b^{-3})\cdot (-4a^{-1}b)\div (12a^{-4}b^{-2}c)\]
\end{exe}
\end{document}
